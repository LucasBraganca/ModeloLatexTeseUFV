% *** Dados ***
\newcommand{\nome}{Dino da Silva Sauro}
\newcommand{\tipo}{Dissertação} %Tese ou dissertação
\newcommand{\titulo}{Titulo do trabalho}
\newcommand{\curso}{Curso}
\newcommand{\programa}{Programa de Pós-Graduação} 
\newcommand{\titulop}{Magister Scientiae} % ou Doctor Scientiae
\newcommand{\cidade}{Viçosa}
\newcommand{\estado}{Minas Gerais}
\newcommand{\mes}{Abril} % Mês letra maiúscula
\newcommand{\ano}{1991}
\newcommand{\aprovacao}{26 de abril de 1991} % mês com letra minúscula formato {dia} 
\newcommand{\instituicao}{Universidade Federal de Viçosa}

% *** Informações para o resumo e abstract ***
\newcommand{\nomecite}{SAURO, Dino da Silva} % Sobrenome em maiúsculo
\newcommand{\titulacao}{M.Sc.} % ou D. sc.
\newcommand{\orientador}{Bradley P. Richfield}
\newcommand{\coorientador}{Roy Hess}
\newcommand{\aprovacaoPT}{fevereiro de 2019} % Mês com leta minúscula
\newcommand{\aprovacaoEng}{February, 2019} % Mês com letra maiúscula
\newcommand{\tituloEng}{Work Title in English}

% As palavras chaves devem ser terminadas com ponto final e separadas por um espaço simples
\newcommand{\keywordsPT}{Palavra chave 1. Palavra chave 2.}
\newcommand{\keywordsEng}{Key word 1. Key word 2.}

% *** Informações para o resumo e abstract ***
% *** Não editar ***
\newcommand{\tituloresumo}{\nomecite, \titulacao, \instituicao, \aprovacaoPT. \textbf{\titulo}. Orientador: \orientador. Coorientador: \coorientador.}
\newcommand{\tituloabstract}{\nomecite, \titulacao, \instituicao, \aprovacaoEng. \textbf{\tituloEng}. Advisor: \orientador. Co-advisor: \coorientador.}
%%%%%%%%%%%%%%%%%%%%%%%%%
\hyphenation{Tes-te sí-la-bas pa-ra-guai}
\newcommand{\fontcapa}{\fontsize{10pt}{11pt}\selectfont}
%%%%%%%%%%%%%%%%%%%%%%%%%%
\newcommand{\resumoPT}{
%*************************
Escreva seu resumo aqui!
%*************************
}
%%%%%%%%%%%%%%%%%%%%%%%%%%
%%%%%%%%%%%%%%%%%%%%%%%%%%
\newcommand{\resumoEng}{
%*************************
Write the abstract here!
%*************************
}
